% #############################################################################
% RESUMO em Português
% !TEX root = ../main.tex
% #############################################################################
% use \noindent in firts paragraph
% reset acronyms
\acresetall

A Arquitetura de Computadores é uma das principais áreas de conhecimento em Engenharia Informática, com unidades curriculares dedicadas incluídas de forma regular nos planos de estudo. Os docentes da área procuram frequentemente ensinar não só os conceitos teóricos, mas também as suas aplicações práticas. Estas tarefas práticas são realizadas maioritariamente em laboratório, recorrendo a simuladores dos sistemas computacionais lecionados. A organização de uma disciplina deste tipo exige decisões quanto aos tópicos a abordar, os exercícios práticos e o trabalho laboratorial. A escolha de que simulador utilizar, ou se deve ser desenvolvido um de raiz, é particularmente importante, pois define a base sobre a qual a maioria do trabalho de laboratório terá de ser construída.

O Ripes é um simulador visual de arquitetura de computadores e editor de código assembly desenvolvido para a arquitetura de instruções (ISA) Reduced Instruction Set Computer V (RISC‑V).  Explora conceitos proeminentes de arquitetura de computadores, como a execução de código máquina, compilação, montagem (assembly) e a interação com dispositivos de Entrada/Saída (IO) mapeados em memória.

Apesar da sua utilidade no apoio a docentes e estudantes, o Ripes não oferece suporte nativo para a execução de instruções de vírgula flutuante (FP).  Estender o simulador nesta direção permite aos estudantes visualizar pipelines de FP, ficheiros de registos separados para FP e de que forma os modos de arredondamento compatíveis com o padrão IEEE 754 influenciam a execução dos programas.

