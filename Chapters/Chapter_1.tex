% #############################################################################
% This is Chapter 1
% !TEX root = ../main.tex
% #############################################################################
% Change the Name of the Chapter i the following line
\fancychapter{Introduction}
%\cleardoublepage
% The following line allows to ref this chapter
\label{chap:intro}
\section{RISC-V}

\ac{RISC-V} is an open-source \ac{ISA} used to develop custom processors for a variety of applications, from embedded designs to supercomputers.
It is considered the fifth generation of processors built on \ac{RISC} and it is suitable for direct native hardware implementation \cite{RISC-V-Unpriv-Intro-2025}.
It was orinaginally developed in 2010 at the University of California, Berkeley and is currently managed by \ac{RISC-V} International, 
which has more than 3000 members which reported that more than 10 bilion chips containing \ac{RISC-V} cores had shipped by the end of 2022 \cite{Synopsys_WhatIsRISCV}.
It's objectives are to provide an open \ac{ISA} that avoids "over-architecting" for particular microarchitectures or implementations while keeping
efficient implementation in any of these.

\section{Visual Tools}

A visual program, often called a \ac{GUI}, is the primary way most users interact with eletronic devices today. Consequently, choosing a visual tool for teaching will certainly lower the barrier of entry 
when beginners first take the challenge of learning \ac{RISC-V}. \acs{GUI}s present information in a more intuitive and visually pleasent manner \cite{Okta2020WhyCLIsSuck}. For these reasons, 
a \ac{GUI} seems the best choice of type of tool to use in computer architecture teaching.

\section{Floating Point}

\acl{FP} is the main method computers use to represent and perform arithmetic on subsets of real numbers that aren't integers. Many scientific, engineering and finantial calculations
require precise calculation of decimal numbers. This is performed accurately and efficiently by \ac{FP} arithmetic. \cite{LenovoFloatingPoint} 

\ac{FP} numbers are typically represented using a standardized format known as \ac{IEEE} 754 \cite{ieee754}. This standard defines formats and methods for binary and decimal \ac{FP} arithmetic
in computer programming environments. It also defines exception conditions and rounding with their default handling. It is able to handle a wide dynamic range that includes tiny atomic sizes
to vast space distances without limiting precision. Therefore, most computer architecture students should learn about \ac{FP} in order to enable real-world applications while avoiding common pitfalls \cite{PythonFloatingPointTutorial} that may cause failures and inprecisions.  

\section{Objectives}

The main objective for this project is to implement the \ac{FP} extensions of the \ac{RISC-V} \ac{ISA} in the single cycle processor in Ripes simulator. 
This will allow teachers from Técnico and around the world who already use Ripes to teach their computer architecture courses to include \ac{FP} in 
their laboratory work by using the same simulator that is already used for the rest of the courses.

The secondary objective is to get this implementation approved by the original developers and integrated in the Ripes original source code, 
so it can a part of the original project and be expanded and maintained by other developers. 


