% #############################################################################
% This is the Glossary Definition List
% !TEX root = ../main.tex
% #############################################################################
% 
%%%%%%%%%%%%%%%%% LIST OF Glossary Terms  %%%%%%%%%%%%%

\newglossaryentry{maths}{%
    name=mathematics,
    description={Mathematics is what mathematicians do}
}

\newglossaryentry{LaTeX}{%
    name=LaTeX,
    description={LaTeX It is a mark up language specially suited for scientific documents as it can correctly format documents with all the typographical rules}
}


\newglossaryentry{formula}{%
    name=formula,
    description={A mathematical expression}
}

\newglossaryentry{XLEN}{
name = XLEN,
description={Term used to describe the size of an address space in \ac{RISC-V}}
}

\newglossaryentry{FLEN}{
name = FLEN,
description={Term used to describe the size of the floating-point address space in \ac{RISC-V}}
}

\newglossaryentry{hart}{
name = Hardware Thread,
description={An hardware thread or \ac{hart} is a single hardware execution context of thread that has its set of registers and runs independently. It does not require an independent instruction fetch unit. Software sees \ac{hart} as basic units of execution.}
}

\newglossaryentry{core}{
name = Core,
description = {A component is termed a core if is capable of independent instruction fetch, decode and execution. It can support multiple \ac{hart} through multithreading. A \ac{RISC-V} core might have additional specialized instruction-set extensions, such as floating-point support.}
}




%%%%%%%%%%%%%%%%% LIST OF SYMBOLS  %%%%%%%%%%%%%
% Here you can define the Symbols used in the document
\newglossaryentry{diam0}{%
  name={\ensuremath{D_0}},
  description={Initial Diameter},
  symbol={\ensuremath{\mu{m}}},
  type=symbols
}

\newglossaryentry{surfarea}{%
    name={\ensuremath{A_s}},
    description={Surface Area},
    symbol={\ensuremath{\mu{m}^2}},
    type=symbols
}
